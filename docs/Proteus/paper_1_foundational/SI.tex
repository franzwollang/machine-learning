\documentclass[11pt]{article}
\usepackage[a4paper,margin=1in]{geometry}
\usepackage{amsmath,amssymb,amsfonts,mathtools}
\usepackage{bm}
\usepackage{hyperref}
\usepackage{algorithm}
\usepackage{algpseudocode}
\hypersetup{colorlinks=true,linkcolor=blue,citecolor=blue,urlcolor=blue}
\title{Supplementary Information for\\ Proteus: Adaptive Scale-Space Analysis for Learning Non-Linear Fuzzy Manifold Memberships}
\author{}
\date{}
\begin{document}
\maketitle

\section*{S1. Initial Dimensionality Reduction}
\textbf{Randomized SVD:} Sample up to 20\,000 rows (or all if small), compute a randomized SVD with sketch width $k_{\mathrm{target}}\approx 256$ and oversampling 20. Determine rank by cumulative energy $\ge 99.9\%$ and/or Levina--Bickel intrinsic dimension; keep $r=\max(r_{0.999},2\,\lceil D_{\mathrm{LB}}\rceil)$ if it yields a $\ge 5\%$ reduction; otherwise skip projection. Optionally append 32 JL vectors to tighten distance preservation.
\[P=V_{1:r}^{\top}\ \Rightarrow\ Z = X\,P\,,\quad Z\in\mathbb R^{N\times r}.\]
\textbf{Rationale:} Cuts ANN costs and per-sample updates by $\times 5$--$\times 40$ on high-$d$ data while retaining \,$\ge 99.9\%$ variance and local distances.

\subsection*{S1.1 Optional Pre-processing via Marginal Gaussianisation}
\textbf{Workflow:} For datasets where individual features exhibit high skew or heavy tails, an optional pre-processing step can significantly accelerate convergence. For each feature $x_j$, its empirical CDF $\hat F_j$ is computed on a data subsample. The features are then transformed via the inverse error function to be marginally Gaussian: $z_j = \sqrt{2}\,\operatorname{erf}^{-1}\!(2\hat F_j(x_j)-1)$.

\textbf{Linearity Test and Model Selection:} After this transform, statistical linearity tests (e.g., Graph-Laplacian Curvature Energy and distance correlation) are run on the $z$-space. If the data is found to be linearly separable to within a given tolerance ($p \ge 0.01$), the expensive fitting of a global \texttt{Glow} model can be skipped entirely for that recursion level, and the algorithm can proceed directly to the geometric audit. This provides a principled mechanism for avoiding unnecessary non-linear transformations.

\section*{S2. First-Principles Derivations}
\subsection*{S2.1 Geometric Grid Spacing}
Normalized responses in scale-space obey $\Delta_{\mathrm{FWHM}}(\log\sigma)\approx 0.586$. If two peaks are separated by less than this, they are not reliably distinguishable. Therefore, choose adjacent scales to differ by $\log\sigma_{j+1}-\log\sigma_j = \Delta \approx \tfrac{1}{2}\,\Delta_{\mathrm{FWHM}}$, giving $\sigma_{j+1}/\sigma_j = e^{\Delta}\approx \sqrt{2}$. Since $\tau=\sigma^2$, the geometric ratio in $\tau$ becomes $r=\tau_{j+1}/\tau_j = (\sigma_{j+1}/\sigma_j)^2 \approx 1/2$. In practice we span both directions, so we use $r=1/\sqrt{2}\approx 0.71$ in $\sigma$ (7--9 points/decade) and map consistently into $\tau$.

\subsection*{S2.2 EWMA Weight from Neighborhood Size}
EWMA recursion $m_{t+1}=(1{-}\alpha)m_t+\alpha x_{t+1}$ has half-life $T_{1/2}=\ln(0.5)/\ln(1{-}\alpha)\approx \ln 2/\alpha$ for small $\alpha$. Set $T_{1/2}{=}k$ (one neighborhood of updates) $\Rightarrow$ $\alpha=\ln 2/k$. This ties estimator memory to the actual neighborhood support.

\subsection*{S2.3 Dual-Rate Motion from Variance Correction}
Under isotropy, variance $\sigma_i^2\approx\mathbb E\lVert x{-}w_i\rVert^2$. A radial shift $\delta r$ changes variance by $\Delta\sigma_i^2 \approx 2\,\sqrt{\sigma_i^2}\,\delta r$. Spreading over $H{=}k$ effective updates gives
\[\eta_{\mathrm{GNG},i} \approx \frac{\Delta\sigma_i^2}{2H\,\sigma_i^2} = \frac{1}{2k}\Big(1-\frac{\sigma_i^2}{\tau}\Big)\,;\qquad \eta_{\mathrm{GNG},i}=\frac{\ln 2}{2k}\Big(1-\frac{\sigma_i^2}{\tau}\Big)\ \text{with half-life normalization}.\]
For centering, with $\delta_{\min}=\kappa(1{-}r)\sqrt{\tau}$ and expected drift per update $\approx\sqrt{\tau}$, set $k\,\eta_{\mathrm{cent}}\,\sqrt{\tau}=\delta_{\min}$ $\Rightarrow$ $\eta_{\mathrm{cent}}=\kappa(1{-}r)/k$. Unless stated otherwise, we use a default of $\kappa=0.5$.

\textbf{Deferred nudge mechanism:} Node positions are updated only when the accumulated nudge $\mathbf{a}_i$ exceeds the resolution threshold $\delta_{\min}$, ensuring mesh changes occur only when meaningfully resolvable at the current scale. This prevents high-frequency noise from destabilizing the mesh while maintaining geometric accuracy.

\subsection*{S2.4 Control-to-Threshold Mapping}
Constrain growth to local intrinsic dimension by mapping optimizer control to thresholds: $\tau_{\mathrm{global}}= -D_{\mathrm{subspace}}\,\log(1{-}s_{\mathrm{control}})$ and $\tau_{\mathrm{local},i}= -d_{\mathrm{final},i}\,\log(1{-}s_{\mathrm{control}})$.

\section*{S3. Statistical Gauntlets}
Unless noted, we adopt significance defaults consistent with Stage 1: Welch's $t$ tests at $p{=}0.05$ (one-tailed where directional) for neighborhood-stability triggers, and Wilson one-sided bounds at 90\% (i.e., $z{=}1.2816$) for link significance comparisons.
\subsection*{S3.1 Link Pruning}
\emph{Power shield:} protect a new link until $n_{\mathrm{eff}}\ge n_{\mathrm{req}}$ from an analytic power formula for proportions. \emph{Relative weakness:} compute Wilson one-sided upper bound $\nu^+$ for link proportion $p$ with count $n$:
\[\nu^+ = \frac{ p + z^2/(2n) + z\sqrt{p(1-p)/n + z^2/(4n^2)} }{1+z^2/n}\,,\ z{=}1.2816.\]
The Wilson score interval is used in place of a standard t-test as it is an "exact" test for binomial proportions that does not assume normality, making it robust and accurate even for the low-hit-count links common in early training. Vote to prune if $\nu^+{<}$ median peer significance; require bilateral agreement from both endpoints.

\subsection*{S3.2 Node Pruning}
\emph{Trigger:} require neighborhood stability (Welch $t$ on $\tilde\rho$). \emph{Fair chance:} Poisson arrival-rate test $\lambda_{\mathrm{est}}=h_i/\text{lifetime} \ge 2/\delta_{\min}$. \emph{Relative:} prune if hits$<0.5\times$ neighbor mean \emph{and} variance $\sigma_i^2 < 0.5\,\tau_{\mathrm{local},i}$. 

The composite rule for node pruning (requiring both low relative hits \emph{and} low local variance) is a critical safeguard. It ensures the system only removes nodes that are both statistically insignificant (low-mass) and located in regions that have already been sufficiently linearized. This prevents the algorithm from mistakenly punching holes in the manifold in areas that are still curved and under active refinement.

\subsection*{S3.3 Node Pruning (Stage 2, simplex-arbitrated)}
\emph{Arbitration by simplex star:} A deletion proposal for node $i$ is decided by a vote over its incident simplexes $\mathcal S(i)$; require majority agreement based on local stability and activity metrics. \emph{Geometric veto:} Always perform a reconstruction error test; if removal of $i$ causes a significant increase in local reconstruction error (tent-pole behavior), veto the deletion. \emph{Rationale:} Stage 2's geometry is mediated by the simplicial complex; arbitration by $\mathcal S(i)$ preserves structural integrity in complex regions.

\section*{S4. Torsion and Shape Quality}
The geometric audit operates on the principle that error responses depend on local rigidity: in \emph{plastic} regions, error produces geometric adjustments; in \emph{rigid} regions, error accumulates as structural stress requiring topological resolution. Let $\bar{\mathbf w}$ and $\bar{\mathbf m}$ be barycenters. With $E=[\mathbf w_{v_1}{-}\bar{\mathbf w}\;\cdots\;\mathbf w_{v_d}{-}\bar{\mathbf w}]$ and $M=[\mathbf m_{v_1}{-}\bar{\mathbf m}\;\cdots\;\mathbf m_{v_d}{-}\bar{\mathbf m}]$, define
\[\Omega_S = M^{\top}E - E^{\top}M\,,\qquad \kappa_S = \lVert \Omega_S \rVert_F\,;\qquad R_S = \kappa_S/\tau^*.\]
\textbf{Torsion Ladder:} \emph{Ignore} if $R_S{<}0.05$ (numerical noise). \emph{Keep as-is} if $0.05\le R_S{<}0.30$ (tolerable curvature). \emph{Geometric fix} if $0.30\le R_S{<}0.60$: perform a torsion-aligned split with explicit placement below. \emph{Non-linear upgrade} if $R_S\ge 0.60$: attach a local mini-NSF (see S7) to the affected patch. If the fitted mini-NSF fails to reduce $R_S$ below $0.60$ on the patch, perform a torsion-aligned split as a fallback.

A split is only permitted if the parent simplex (or node) has accumulated sufficient hits to pass a budget check ($h_{\mathrm{parent}} \ge 2 \cdot h_{\mathrm{prune}}$), ensuring that the resulting children are not immediately vulnerable to pruning under the framework's statistical gauntlets.

\emph{Split placement:} Insert the new node at the midpoint of the simplex's longest edge projected onto the principal axis of $\Omega_S$ (dominant torsion axis). If any child has $Q_S{<}0.25$, redo via centroid star split.

\textbf{Shape quality safeguard:} The simplex shape metric $Q_S = d\,r_{\mathrm{in}}/R_{\mathrm{circ}} \in (0,1]$ uses the inradius $r_{\mathrm{in}}$ and circumradius $R_{\mathrm{circ}}$. Poor quality ($Q_S < 0.25$) indicates numerical instability risk; the centroid split fallback ensures stable face normals and well-conditioned gradients.

\section*{S5. Dual Flow Mechanisms}
Accumulate facet pressures for winning simplex $S$ via $\Delta p_f \propto \max(0,(\mathbf x{-}\bar{\mathbf w}_S)^{\top}\mathbf n_f)$. This \emph{fractional tallying} avoids winner–take–all artifacts and yields smoother, more stable boundary pressure estimates than single-face voting. On the dual graph, solve
\[\min_{\{p_f\}}\; \lambda\sum_f (p_f{-}\hat p_f)^2 + \mu\sum_S \lVert A_S\,\mathbf p_S\rVert_2^2\,,\]
with Gauss--Seidel iterations $p_f\leftarrow (\lambda\hat p_f + \mu\,\sum_{S\ni f}(A_S^{\top}A_S\,\mathbf p_S)_f)/(\lambda + \mu\,\deg(f))$, or equivalently by loopy BP messages until convergence.

\section*{S6. Masking for Missing Modalities}
For input $\mathbf x\in\mathbb R^d$ with structured sparsity, build boolean mask $\mathbf m=(x\neq 0)$. Compute distances and residuals only on active dimensions: $\lVert \mathbf x{-}\mathbf w\rVert^2=\sum_{j:m_j} (x_j{-}w_j)^2$; residual $\mathbf e=\mathbf x{-}\mathbf w$ with $e_j{=}0$ if $m_j{=}0$. Apply the mask to all vector updates (moments, nudges, Oja), and accumulate variance only on active dimensions. Optionally normalize updates by the active count to equalize per-sample influence. This preserves geometry and statistics in multi-modal fusion where zeros indicate missingness rather than small values.

\section*{S7. Non-Linear Warp Strategies}
When geometric refinement (node moves and splits) is insufficient to resolve high residual curvature, Proteus can apply a non-linear warp. The core strategy is a hybrid approach that adapts to the nature of the curvature, measured by the torsion coverage $P_\kappa$ (the fraction of simplices with torsion ratio $R_S \ge 0.30$).

\paragraph{Global vs. Patch-wise Strategy:}
\begin{itemize}
  \item If curvature is \textbf{widespread ($P_\kappa > 50\%$)}, a shallow \emph{global Glow} is trained. Its invertible 1$\times$1 convolutions mix dimensions efficiently, providing a good baseline correction for system-wide bends.
  \item If curvature is \textbf{patchy ($P_\kappa \le 25\%$)}, compact and expressive \emph{mini-Normalizing Spline Flows (NSFs)} are attached only to the highest-torsion patches. This avoids wasting model capacity on regions that are already linear.
  \item If curvature is \textbf{mixed ($25\% < P_\kappa \le 50\%$)}, a moderate global Glow is combined with mini-NSFs on the worst $\sim$10\% of patches.
\end{itemize}

\paragraph{Patch Identification and Mini-NSF Training:}
A facet-adjacency graph is built over all "hot" simplices ($R_S \ge 0.60$). Leiden community detection finds contiguous patches of high torsion, with resolution auto-tuned to ensure no patch exceeds a size cap $P_{\max}$ (e.g., 64 simplices). Each mini-NSF is a compact model (2-3 layers of rational-quadratic spline couplings, 64-128 hidden units, 8-12 bins) trained for a small number of epochs on its routed data, with the global model frozen. If, after training, any simplex in the patch still exhibits $R_S \ge 0.60$, attempt a torsion-aligned split as a geometric fallback.

\section*{S8. Numerical Stability and Implementation}
\textbf{Warm-start and state reset:} On entering Stage 2 at fixed $\tau^*$, shrink EWMA moments by a decay factor $\gamma_0\approx 0.7$, carry hit/link counts unchanged, and reset the nudge accumulator $\mathbf a$ to zero to ensure stable dynamics under the richer neighborhood model ($k\approx d$).
\textbf{Conditioning for $\Omega_S$:} Use QR (preferred) or thin SVD to stabilize $E$; scale residuals $M$ by a local norm to avoid magnitude blow-up; compute $\Omega_S=M^TE{-}E^TM$ from stabilized factors; cache per-simplex factors and recompute only after geometric moves/splits.

\textbf{Small constants and clipping:} Set $\varepsilon$ in $\rho_i$ to $10^{-8}$\,–\,$10^{-6}$ depending on data scale; clip $\eta_{\mathrm{GNG}}$ to $\lvert\eta\rvert\le 0.3$ (Stage 1) and $\le 0.05$ (Stage 2); constrain spline slopes/bins to avoid near-singular Jacobians; bound $\lVert\mathbf a_i\rVert$ growth between apply steps.

\textbf{Cache face normals:} Precompute and cache per–simplex face normals; recompute only after deferred moves or splits affecting a vertex. This keeps torsion and face-pressure updates $\mathcal O(d^2)$ and stable.

\textbf{ANN configuration (HNSW):} Typical defaults: $M{=}16$\,–\,32, $\mathrm{efConstruction}{=}100$\,–\,400, $\mathrm{efSearch}{=}50$\,–\,200; increase efSearch modestly near convergence to improve neighbor stability. Persist indices per recursion level; rebuild only after significant mesh mutation.

\subsection*{S8.1 Practical Implementation Notes}
\paragraph{Rank-1 Oja vs. Scalar Variance.} While the core theory holds with a simple scalar variance tracker, the specified implementation uses a rank-1 Oja's rule update to track the dominant principal component of the local error. This provides a more accurate, data-driven vector for placing new nodes during a split, at the cost of one extra stored vector per node. For ambient dimensions $d \lesssim 10,000$, the improved split quality justifies the modest increase in memory, and this method is recommended.

\paragraph{Initial Simplex Seeding.} Upon entering Stage 2, the initial simplicial complex is seeded from the Stage 1 node-edge graph using a fast, "Greedy Chaining" heuristic. This heuristic rapidly discovers the majority of $d$-simplices, providing a robust starting point for the dynamic refinement process.

\subsection*{S8.2 Cost Envelopes}
Stage~1 per-epoch time is $\mathcal O\bigl(N\,d\,k\,\log N_{\mathrm{nodes}}\bigr)$ with HNSW queries; space is $\mathcal O(N_{\mathrm{nodes}}\,d + N_{\mathrm{nodes}}\,\bar d)$. Stage~2 initial simplex seeding by Greedy Chaining is $\mathcal O\bigl(N_c\,d_{\mathrm{cand}}^{\,2}\,d_c\bigr)$; torsion audits are $\mathcal O\bigl(N_{\mathrm{simplices}}\,(d{+}1)^2\,d\bigr)$. The dual graph used by the belief propagation solver has size $\mathcal O(N_{\mathrm{simplices}}\,d)$.

\section*{S9. Robustness and Edge Cases}
\textbf{Outliers:} Use Huber-like weighting on residuals in moment updates or winsorize top 0.5\% residual norms; optionally cap per-sample hit contributions.

\textbf{Boundaries and orientation:} For open meshes, treat boundary faces with Neumann-like conditions in the dual flow (no flux across exterior). In non-orientable regions, operate on local orientation patches to keep face normals consistent.

\textbf{Sparse patches:} Require a minimum routed sample count $n_{\min}=\max(10\,d, 1000)$ before fitting a mini-NSF; otherwise defer to geometric refinement.

\section*{S10. Evaluation Protocols}
\textbf{Defaults:} Stage 1 $k{=}8$, $\alpha{=}\ln 2/8$, grid $r{=}1/\sqrt{2}$; Stage 2 $k{\approx}d$, $\alpha{=}\ln 2/k$. Stopping: CV$<0.01$ and $R_S{<}0.30$. PH: VR complex, max dim 2, filtration up to $1.5\,\sigma_\star$. MMD: RBF kernel, bandwidth median heuristic, 1\,000 samples/1\,000 tests. Ablations: remove Stage 1 scaffold, Torsion Ladder, or Dual Flow.

\textbf{Persistent Homology:} Vietoris–Rips (max dim 2), filtration to $1.5\,\sigma_\star$; report bottleneck/2-Wasserstein.
\textbf{MMD:} RBF kernel (median heuristic), 1000 samples/1000 tests, 5 seeds.
\textbf{Log-likelihood:} Sum of face-pressure potentials within simplex plus normalization; use held-out data.

\section*{S11. Algorithmic Details}
The following procedures reference node and link bookkeeping commonly used in the implementation: nodes track an \emph{update count} and creation index $N_{\mathrm{creation}}$ in addition to positions and EWMA statistics; links maintain directed counters and recent activity snapshots. These fields support pruning power analyses and tenure checks but are not essential for the mathematical description.

\paragraph{Flow-conserving initialization after a split.}
Let $C$ be the parent simplex with vertices $V(C)=\{v_1,\dots,v_{d{+}1}\}$ and let $h_{\mathrm{old}}(v_i,v_j)$ denote pre-split internal hit counts between vertices. When inserting a new vertex $k$, initialize the new tallies by
\[
  h_{\mathrm{new}}(k,v_i) \;=\; \tfrac{1}{2}\sum_{\substack{v_j\in V(C)\\ j\ne i}} h_{\mathrm{old}}(v_i,v_j)\,.
\]
This preserves the total internal mass $H_{\mathrm{total}}(C)$ and approximately preserves the local flow gradient.

\paragraph{Data structures (Stage 2 additions).} Each simplex maintains small fields to support the torsion audit and dual-flow:
\begin{itemize}
  \item \texttt{face\_pressures}: array of size $d{+}1$ for facet pressures
  \item \texttt{torsion\_sum}, \texttt{torsion\_sum\_sq}: accumulators for $\kappa_S$
  \item \texttt{simplex\_update\_count}: number of updates contributing to the audit
\end{itemize}
\begin{algorithm}[h]
\caption{Stage 1 at Fixed $\tau$}
\label{alg:stage1}
\begin{algorithmic}[1]
\Procedure{RunToEquilibrium}{$X, \tau$}
    \State Initialize mesh $\mathcal M \leftarrow \varnothing$ \Comment{Create links between BMU$_1$ and BMU$_2$ on first co-activation; maintain directed weighted counters $C_{12},C_{21}$}
    \Repeat
        \State Sample $\mathbf x$ from $X$
        \State $\mathcal{N}_k \gets \text{QueryANN}(\mathbf x, k)$ \Comment{Find $k$ nearest nodes}
        \State \text{ApplyRankWeightedUpdates}($\mathbf x, \, \mathcal{M}, \, \mathcal{N}_k$)
        \For{\textbf{each} updated node $i$}
            \If{$\lVert \mathbf{node}_i.\mathbf a \rVert \ge \delta_{min}$} \Comment{Deferred move}
                \State $\mathbf{node}_i.\mathbf w \mathrel{+}= \mathbf{node}_i.\mathbf a$; $\mathbf{node}_i.\mathbf a \gets 0$; $\text{RefreshANN}(\mathbf{node}_i)$
            \EndIf
            \If{$\mathbf{node}_i.\sigma^2 > \tau_{local,i}$} \Comment{Split rule}
                \State $\text{SplitNode}(\mathbf{node}_i)$ \Comment{Flush nudge; place child at $\mathbf w_i{+}\mathbf m_i$ along $\mathbf u_i$; shrink-inherit stats}
            \EndIf
        \EndFor
        \State $\text{ApplyPruningGauntlets}(\mathcal{M})$ \Comment{Per SI S3}
    \Until{$\text{ComputeCV}(\mathcal M) < 0.01$}
    \State \textbf{return} $\mathcal M$
\EndProcedure
\end{algorithmic}
\end{algorithm}

\begin{algorithm}[h]
\caption{Rank-weighted neighbor updates}
\label{alg:rankupdates}
\begin{algorithmic}[1]
\Procedure{ApplyRankWeightedUpdates}{$\mathbf x, \mathcal{M}, \mathcal{N}_k$}
    \For{$j=1 \dots k$ \textbf{in} $\mathcal{N}_k$ (rank-ordered)}
        \State $w_j \gets 2^{-(j-1)}$
        \State $\mathbf{node} \gets \mathcal{N}_k[j]$
        \State $\mathbf e \gets \mathbf x - \mathbf{node}.\mathbf w$
        \State $\mathbf{node.m} \gets (1-\alpha w_j)\mathbf{node.m} + \alpha w_j\,\mathbf e$; $\mathbf{node.s} \gets (1-\alpha w_j)\mathbf{node.s} + \alpha w_j\,\mathbf e^{\circ 2}$
        \State $\rho \gets \lVert \mathbf{node.m} \rVert/(\mathbf{node.}\sigma+\varepsilon)$; $\mathbf{node.a} \gets \mathbf{node.a} + \eta_{cent}\,\rho\,\mathbf{node.m}$
        \State $\text{UpdateOja}(\mathbf{node.u}, \mathbf e)$
        \State $\mathbf{node.h} \gets \mathbf{node.h} + w_j$ \Comment{Fractional hits and link counters}
    \EndFor
\EndProcedure
\end{algorithmic}
\end{algorithm}

\begin{algorithm}[h]
\caption{Simplex-native updates (Stage 2)}
\label{alg:stage2updates}
\begin{algorithmic}[1]
\Procedure{ApplySimplexNativeUpdates}{$\mathbf x, \mathcal{C}$}
    \State $C \gets \text{LocateWinningSimplex}(\mathbf x)$ \Comment{Barycentric or nearest with projection}
    \State Order vertices of $C$ by increasing $\lVert \mathbf x - \mathbf w\_v \rVert$; assign rank-weights $w_j=2^{-(j-1)}$
    \For{each vertex $v$ of $C$ in rank order}
        \State $\mathbf e \gets \mathbf x - \mathbf w\_v$; $\mathbf m\_v \gets (1-\alpha w_j)\mathbf m\_v + \alpha w_j\,\mathbf e$; $\mathbf s\_v \gets (1-\alpha w_j)\mathbf s\_v + \alpha w_j\,\mathbf e^{\circ 2}$
        \State $\rho \gets \lVert \mathbf m\_v \rVert/(\sigma\_v+\varepsilon)$; $\mathbf a\_v \gets \mathbf a\_v + \eta\_{cent}\,\rho\,\mathbf m\_v$
    \EndFor
    \State Accumulate facet pressures for $C$ per S5
\EndProcedure
\end{algorithmic}
\end{algorithm}

\section*{S12. Theoretical Guarantees}
\subsection*{S12.1 Expressivity and Approximation}
Let $p$ be a probability density on a compact set $\Omega\subset\mathbb R^d$ with a (piecewise) Lipschitz continuous log-density and bounded curvature of its principal level sets. For any $\varepsilon{>}0$, there exist finite budgets for recursion depth, mesh resolution, and non-linear warps (mini-NSFs) such that the model density $q$ learned by the Proteus procedure satisfies $\lVert p-q\rVert_{L_1} < \varepsilon$.

The proof is constructive, following the algorithmic stages. We show that the total approximation error can be bounded by the sum of errors from a hierarchical decomposition, a piecewise-linear approximation of each component, and a non-linear residual correction. By making the resolution and capacity of each stage sufficient, the total error can be made arbitrarily small.

\subsection*{S12.2 Correspondence: Simplex Equilibrium Implies Effective Node Equilibrium}
\paragraph{Setting and assumptions.} Consider a simplicial complex with nondegenerate simplex stars around each node, locally stationary sampling, and a fixed characteristic scale $\tau^*$. Let $P(V_i)$ denote Voronoi mass at node $i$ and $P(C)$ the mass of simplex $C$.

\paragraph{Claim.} Driving the system toward simplex equilibrium ($P(C) \approx 1/M$ for all simplexes) induces effective node equilibrium ($P(V_i) \approx 1/N$ up to junction effects and sampling noise).

\paragraph{Sketch.} Suppose a node $i$ has excess mass $P(V_i) \gg 1/N$. Samples concentrate near $w_i$, biasing residuals in the adjacent simplex star $\mathcal S(i)$. The barycentric residual means in $\mathcal S(i)$ acquire outward components, raising torsional stress and face pressures that cannot be canceled by PL translation alone. The Torsion Ladder therefore either (i) performs a torsion-aligned split creating a child $j$ in the outward direction, or (ii) fits a local mini-NSF that unbends the region and reduces the asymmetric flux. In both cases the effective domain of $V_i$ contracts in favor of $V_j$ or of neighboring stars, strictly decreasing $P(V_i)$ while increasing mass in adjacent cells. Repeating this argument shows that any persistent Voronoi over-mass produces simplex-level stresses whose resolution decreases the imbalance, establishing a negative feedback toward node equilibrium. Deviations persist only at structural junctions (e.g., different intrinsic dimensions), where systematic bimodality and asymmetric links form an informative signature rather than a failure.

\paragraph{Remarks.} The argument uses only local stationarity and nondegenerate stars; concentration bounds convert it to a finite-sample statement. Empirically, the correspondence manifests around overloaded nodes as: sharply bimodal simplex activity within the node's star at dimensionality junctions; a clean partition of link statistics into high-significance backbone links versus low-significance bridge links; and bridge links that exhibit persistent directional hit asymmetry together with low net face pressure, distinguishing stable junctions from active frontiers.

\paragraph{The Signature of a Dimensionality Junction.} The "informative mismatch" at structural junctions manifests as a concrete, stable, and observable statistical signature. A node located where, for example, a 1D filament meets a 2D sheet will persistently exhibit:
\begin{itemize}
    \item \textbf{Bimodal Simplex Activity:} The distribution of internal activity values for its adjacent simplexes will be sharply bimodal.
    \item \textbf{Divergent Link Significance:} Its links will partition into high-significance "backbone" links (within a manifold) and low-significance "bridge" links (connecting manifolds).
    \item \textbf{Stable Asymmetry on Bridge Links:} Bridge links show high, stable directional hit-counter asymmetry but, crucially, low net face pressure. This distinguishes a stable junction from an active, growing frontier, where both asymmetry and pressure would be high.
\end{itemize}
This emergent signature is not a failure of the algorithm, but a feature that allows the system to report on the complex local topology of the data.

\end{document}
